% arara: xelatex
\documentclass[12pt]{article} % размер шрифта

\usepackage{tikz} % картинки в tikz
\usepackage{microtype} % свешивание пунктуации

\usepackage{array} % для столбцов фиксированной ширины

\usepackage{url} % для вставки ссылок \url{...}

\usepackage{indentfirst} % отступ в первом параграфе

\usepackage{sectsty} % для центрирования названий частей
\allsectionsfont{\centering} % приказываем центрировать все sections

\usepackage{amsthm} % теоремы и доказательства

\usepackage{physics}

\usepackage{amsmath, amssymb} % куча стандартных математических плюшек

\usepackage[top=2cm, left=1.5cm, right=1.5cm, bottom=2cm]{geometry} % размер текста на странице

\usepackage{lastpage} % чтобы узнать номер последней страницы

\usepackage{enumitem} % дополнительные плюшки для списков
%  например \begin{enumerate}[resume] позволяет продолжить нумерацию в новом списке
\usepackage{caption} % подписи к картинкам без плавающего окружения figure


\usepackage{fancyhdr} % весёлые колонтитулы
\pagestyle{fancy}
\lhead{ARMA: no nonsense notes}
\chead{}
\rhead{}
\lfoot{}
\cfoot{}
\rfoot{\thepage}
\renewcommand{\headrulewidth}{0.4pt}
\renewcommand{\footrulewidth}{0.4pt}



\usepackage{todonotes} % для вставки в документ заметок о том, что осталось сделать
% \todo{Здесь надо коэффициенты исправить}
% \missingfigure{Здесь будет картина Последний день Помпеи}
% команда \listoftodos — печатает все поставленные \todo'шки

\usepackage{booktabs} % красивые таблицы
% заповеди из документации:
% 1. Не используйте вертикальные линии
% 2. Не используйте двойные линии
% 3. Единицы измерения помещайте в шапку таблицы
% 4. Не сокращайте .1 вместо 0.1
% 5. Повторяющееся значение повторяйте, а не говорите "то же"

\usepackage{fontspec} % поддержка разных шрифтов
\usepackage{polyglossia} % поддержка разных языков

\setmainlanguage{english}
% \setotherlanguages{russian}

\setmainfont{Linux Libertine O} % выбираем шрифт
% если Linux Libertine не установлен, то
% можно также попробовать Helvetica, Arial, Cambria и т.Д.

% чтобы использовать шрифт Linux Libertine на личном компе,
% его надо предварительно скачать по ссылке
% http://www.linuxlibertine.org/index.php?id=91&L=1

% на сервисах типа sharelatex.com этот шрифт есть :)

\newfontfamily{\cyrillicfonttt}{Linux Libertine O}
% пояснение зачем нужно шаманство с \newfontfamily
% http://tex.stackexchange.com/questions/91507/

\AddEnumerateCounter{\asbuk}{\russian@alph}{щ} % для списков с русскими буквами
%\setlist[enumerate, 2]{label=\asbuk\cdot),ref=\asbuk\cdot} % списки уровня 2 будут буквами а) б) ...

%% эконометрические и вероятностные сокращения
\DeclareMathOperator{\Cov}{Cov}
\DeclareMathOperator{\Corr}{Corr}
\DeclareMathOperator{\Var}{Var}
\DeclareMathOperator{\E}{E}
\def \hb{\hat{\beta}}
\def \hs{\hat{\sigma}}
\def \htheta{\hat{\theta}}
\def \s{\sigma}
\def \hy{\hat{y}}
\def \hY{\hat{Y}}
\def \v1{\vec{1}}
\def \e{\varepsilon}
\def \he{\hat{\e}}
\def \z{z}
\def \hVar{\widehat{\Var}}
\def \hCorr{\widehat{\Corr}}
\def \hCov{\widehat{\Cov}}
\def \cN{\mathcal{N}}
\def \RR{\mathbb{R}}
\def \ZZ{\mathbb{Z}}


\usepackage{thmtools} % тонкая настройка окружения теорем

% окружения!
\declaretheorem[
style=definition,
title=Theorem,
numberwithin=,
]{theorem}


% опция sibling=theorem посередине даёт сквозную нумерацию с теоремами
\declaretheorem[
style=definition,
title=Definition,
sibling=theorem,
qed={}
]{definition}


% опция sibling=theorem посередине даёт сквозную нумерацию с теоремами
\declaretheorem[
style=definition,
title=Example,
sibling=theorem,
qed={}
]{example}





\begin{document}

Short introduction to ARMA procesess without nonsense. 
The goal is to state all the theorems rigorously. 

\begin{definition}
The process $(u_t)$ is called white noise if 
\[
\E(u_t) = 0, \quad \Var(u_t) = \sigma^2, \quad \Cov(u_s, u_t) = 0 \text{ for } s\neq t.    
\]
\end{definition}

This definition does not assume that $u_t$ and $u_s$ are independent. 
They may be dependent but uncorrelated. 

This definition does not assume normality of $u_t$ but normality of white noise 
is often assumed in maximum likelihood estimation. 

\begin{definition}
    Lag operator $L$ transforms a stochastic process $(y_t)$ with $t\in \ZZ$ into a 
    new stochastic process by shifting the index back in time,
    \[
            L y_t = y_{t-1}.
    \]
\end{definition}

\begin{definition}
    Forward operator $F$ transforms a stochastic process $(y_t)$ with $t\in \ZZ$ into a 
    new stochastic process by shifting the index forward in time,
    \[
            F y_t = y_{t+1}.
    \]
\end{definition}


Simple arithmetic examples are:
\[
(1 + 2L + 3L^2)y_t = y_t + 2y_{t-1} + 3y_{t-2},   
\]
\[
(3 + 2F + 5F^2)y_t = 3y_t + 2y_{t+1} + 5y_{t+2},
\]

\begin{theorem}
    The operators $L$ and $F$ are linear and $L^{-1} = F$.
\end{theorem}

\begin{proof}
 The action $LF$ or $FL$ does nothing with any process $(y_t)$. 
 So operators $L$ and $F$ are mutually inverse.
\end{proof}


\begin{definition}
    The process $(y_t)$ is called stationary in weak sense if 
    \[
    \E(y_t) = \mu, \quad \Cov(u_s, u_t) = \gamma(t - s).    
    \]
\end{definition}

In particular all variances of stationary process are equal, $\Var(y_t) = \Cov(y_t, y_t) = \gamma_0$.



When infinite sums do exist?


...



We \textit{define} division by monomials.



\begin{definition}
    For $\abs{\alpha} < 1$ we define
    \[
    \frac{1}{1 - \alpha L} y_t = (1 + \alpha L + \alpha^2 L^2 +\alpha^3 L^3 + \ldots )y_t,    
    \]
    and 
    \[
    \frac{1}{1 - \alpha F} y_t = (1 + \alpha F + \alpha^2 F^2 +\alpha^3 F^3 + \ldots )y_t.    
    \]
\end{definition}


\begin{theorem}
    If $(u_t)$ is a white noise and $\abs{\alpha} < 1$ then 
    $\frac{1- \alpha L}{1 - \alpha F} u_t$ and $\frac{1- \alpha F}{1 - \alpha L} u_t$ are white noises. 
\end{theorem}



\begin{theorem}
    The equation for $(y_t)$
    \[
    P(L) y_t = Q(L)u_t + c,    
    \]
    where $(u_t)$ is a white noise 
    has infinitely many non-stationary solutions $(y_t)$ if degree of $P$ is higher than one. 
\end{theorem}

\begin{theorem} \label{thm:solutionstructure}
Consider the equation for $(y_t)$
\[
    P(L) y_t = Q(L)u_t + c,
\]
where $(u_t)$ is a white noise.
If polynomials $P$ and $Q$ are coprime then 
\begin{enumerate}
    \item There are no stationary solutions $(y_t)$ at all if $P$ has at least one root $\ell$ with $\abs{\ell} = 1$.
    \item There is exactly one stationary solution $(y_t)$ if all roots $\ell$ of $P$ have $\abs{\ell} \neq 1$.
\end{enumerate}

There are two subcases when all roots $\ell$ of $P$ have $\abs{\ell} \neq 1$:
\begin{enumerate}
    \item All roots $\ell$ of $P$ have $\abs{\ell} > 1$.
    In this case the unique stationary solution has the form 
    \[
        y_t = \mu + u_t + c_1 u_{t-1} + c_2 u_{t-2} + c_3 u_{t-3} + \ldots,
    \]
    where $(u_t)$ is the white noise from original equation. 
    \item At least one root of $P$ has $\abs{\ell} <1$.
    In this case the unique stationary solution has the form 
    \[
        y_t = \mu + \nu_t + c_1 \nu_{t-1} + c_2 \nu_{t-2} + c_3 \nu_{t-3} + \ldots,
    \]
    where $(\nu_t)$ is a white noise different from $(u_t)$.

\end{enumerate}

\end{theorem}








\begin{definition}
    The process $(y_t)$ is called $ARMA(p, q)$ process with equation  
    \[
    P(L) y_t = Q(L) u_t + c,    
    \]
    if 
    \begin{enumerate}
        \item the process $(y_t)$ satisfies this equation;
        \item polynomial $P(L)$ has degree $p$ and polynomial $Q(L)$ has degree $q$;
        \item $P(0) = Q(0) = 1$;
        \item $P$ and $Q$ are coprime, in other words they have no common roots. 
        \item the process $(y_t)$ can be represented in $MA(\infty)$ form with respect to $(u_t)$:
        \[
            y_t = \mu + u_t + c_1 u_{t-1} + c_2 u_{t-2} + c_3 u_{t-3} + \ldots
        \]
    \end{enumerate}
\end{definition}


From the last requirement in this definition it \textit{follows} that all $ARMA(p,q)$ processes are 
stationary. By definition. Point. By the theorem \ref{thm:solutionstructure} the polynomial $P$ has all roots with $\abs{\ell} > 1$.
The equation $P(L) y_t = Q(L) u_t + c$ has infinitely many non stationary solutions. 
Not all solutions of equation $P(L) y_t = Q(L) u_t + c$ are called $ARMA$ processes. 
The same process $y_t$ can have \textit{many} $ARMA(p, q)$ representations with the same $AR$ part and 
different $MA$ parts. 

In these examples $(u_t)$ is a white noise. 

\begin{enumerate}
    \item Equation $y_t = 5 + y_{t-1} + u_t$.
    This equation has no stationary solutions at all. 
    This equation has infinite number of non-stationary solution. 

    Polynomials $P$ and $Q$ are coprime, unit root in the polynomial $P$. 
    We do not use this equation to describe $ARMA$ process. 

    \item Equation $y_t = y_{t-1} + u_t - u_{t-1}$.
    This equation has infinite number of stationary solutions. For example $y_t = u_t + 5$.
    This equation has infinite number of non-stationary solution. 
    
    Polynomials $P$ and $Q$ are not coprime, unit root in both polynomials $P$ and $Q$.
    We do not use this equation to describe $ARMA$ process. 

    \item Equation $y_t = 5 + 2 y_{t-1} + u_t - u_{t-1}$.
    This equation has unique stationary solution. 
    This equation has infinite number of non-stationary solution. 
    
    Polynomials $P$ and $Q$ are coprime, not all roots of $P$ are outside unit circle. 

    We do not use this equation to describe $ARMA$ process. 

    \item Equation $y_t = 7 + 0.5 y_{t-1} + u_t - u_{t-1}$.
    This equation has unique stationary solution. 
    This equation has infinite number of non-stationary solution. 
    
    Polynomials $P$ and $Q$ are coprime, all roots of $P$ are outside unit circle. 

    The stationary solution of this equation is called $ARMA(1, 1)$ process with equation $y_t = 0.5 y_{t-1} + u_t - u_{t-1}$.


\end{enumerate}



\begin{definition}
    The $ARMA(p, q)$ process with \textit{equation}
    \[
    P(L) y_t = Q(L) u_t + c,    
    \]
    is called \textit{invertible} if white noise values can be recovered from past observed $(y_t)$ in a linear form,
    \[
       u_t = b + y_t + d_1 y_{t-1} + d_2 y_{t-2} + d_3 y_{t-3} + \ldots      
    \]
\end{definition}


Stationarity is the property of the process per se, invertibility is the property of \textit{the process and the equation}. 
One cannot check whether a given sequence of random variables is invertible. 


\begin{example}
Consider two white noise processes $(u_t)$ and $(v_t)$ linked by equation $(1 - 2L) u_t = (1 - 0.5 L)v_t$.
This equation may be solved to obtain $u_t = \frac{1-0.5L}{1-0.5F} \frac{F}{-2} v_t$, that shows that the two white 
noises do exist. 

Now consider the process $y_t = (1 - 2L) u_t = (1 - 0.5 L)v_t$.

Is $(y_t)$ invertible? This question is meaningless. 

The process $(y_t)$ described by equation $y_t = u_t - 2u_{t-1}$ is not invertible. 

The same process $(y_t)$ described by equation $y_t = v_t - 0.5v_{t-1}$ is invertible. 
\end{example}




\begin{theorem}
Any $ARMA(p, q)$ process has at most one invertible equation.
\end{theorem}

\begin{theorem}
    The $ARMA(p, q)$ process with \textit{equation}
    \[
    P(L) y_t = Q(L) u_t + c,    
    \]
    is \textit{invertible} if and only if all roots $\ell$ of $P$ have $\abs{\ell} > 1$.
\end{theorem}

By our definition of $ARMA$ process, it is stationary, polynomials $P$ and $Q$ are coprime and $P$ has all roots with $\abs{\ell} > 1$.






\end{document}
