\documentclass[12pt]{article}

\usepackage{physics}


\usepackage{tikz} % картинки в tikz
\usepackage{microtype} % свешивание пунктуации

\usepackage{array} % для столбцов фиксированной ширины

\usepackage{indentfirst} % отступ в первом параграфе

\usepackage{sectsty} % для центрирования названий частей
\allsectionsfont{\centering}

\usepackage{amsmath, amsfonts, amssymb} % куча стандартных математических плюшек

\usepackage{comment}

\usepackage[top=2cm, left=1.2cm, right=1.2cm, bottom=2cm]{geometry} % размер текста на странице

\usepackage{lastpage} % чтобы узнать номер последней страницы

\usepackage{enumitem} % дополнительные плюшки для списков
%  например \begin{enumerate}[resume] позволяет продолжить нумерацию в новом списке
\usepackage{caption}

\usepackage{url} % to use \url{link to web}

\usepackage{fancyhdr} % весёлые колонтитулы
\pagestyle{fancy}
\lhead{Time Series and Stochastic Processes}
\chead{}
\rhead{Midterm, 2021-10-28}
\lfoot{2021-2022}
\cfoot{}
\rfoot{\thepage/\pageref{LastPage}}
\renewcommand{\headrulewidth}{0.4pt}
\renewcommand{\footrulewidth}{0.4pt}

\usepackage{tcolorbox} % рамочки!

\usepackage{todonotes} % для вставки в документ заметок о том, что осталось сделать
% \todo{Здесь надо коэффициенты исправить}
% \missingfigure{Здесь будет Последний день Помпеи}
% \listoftodos - печатает все поставленные \todo'шки


% более красивые таблицы
\usepackage{booktabs}
% заповеди из докупентации:
% 1. Не используйте вертикальные линни
% 2. Не используйте двойные линии
% 3. Единицы измерения - в шапку таблицы
% 4. Не сокращайте .1 вместо 0.1
% 5. Повторяющееся значение повторяйте, а не говорите "то же"



\usepackage{fontspec}
\usepackage{polyglossia}

\setmainlanguage{english}
\setotherlanguages{english}

% download "Linux Libertine" fonts:
% http://www.linuxlibertine.org/index.php?id=91&L=1
\setmainfont{Linux Libertine O} % or Helvetica, Arial, Cambria
% why do we need \newfontfamily:
% http://tex.stackexchange.com/questions/91507/
\newfontfamily{\cyrillicfonttt}{Linux Libertine O}

%\AddEnumerateCounter{\asbuk}{\russian@alph}{щ} % для списков с русскими буквами
%\setlist[enumerate, 2]{label=\asbuk*),ref=\asbuk*}

%% эконометрические сокращения
\DeclareMathOperator{\Cov}{\mathbb{C}ov}
\DeclareMathOperator{\Corr}{\mathbb{C}orr}
\DeclareMathOperator{\Var}{\mathbb{V}ar}

\let\P\relax
\DeclareMathOperator{\P}{\mathbb{P}}

\DeclareMathOperator{\E}{\mathbb{E}}
% \DeclareMathOperator{\tr}{trace}
\DeclareMathOperator{\card}{card}
\DeclareMathOperator{\plim}{plim}
\DeclareMathOperator{\pCorr}{\mathrm{p}\mathbb{C}\mathrm{orr}}


\newcommand \hb{\hat{\beta}}
\newcommand \hs{\hat{\sigma}}
\newcommand \htheta{\hat{\theta}}
\newcommand \s{\sigma}
\newcommand \hy{\hat{y}}
\newcommand \hY{\hat{Y}}
\newcommand \e{\varepsilon}
\newcommand \he{\hat{\e}}
\newcommand \z{z}
\newcommand \hVar{\widehat{\Var}}
\newcommand \hCorr{\widehat{\Corr}}
\newcommand \hCov{\widehat{\Cov}}
\newcommand \cN{\mathcal{N}}
\newcommand \RR{\mathbb{R}}
\newcommand \NN{\mathbb{N}}
\newcommand{\cF}{\mathcal{F}}
\newcommand{\cH}{\mathcal{H}}


\begin{document}

Short rules: 120 minutes, online without proctoring. You may use any source you want but don't cheat.
\vspace{10pt}

\textbf{Pledge:}
\vspace{10pt}
\begin{tcolorbox}
Start exam by writing the following honor pledge and signing it.

\vspace{10pt}

I pledge on my honor that I will not give nor receive any 
unauthorized assistance on this exam.
\end{tcolorbox}
\vspace{20pt}

\textbf{Problems:}


\begin{enumerate}

\item (10 points) Consider the Markov chain with the transition matrix
\[
  P = \begin{pmatrix}
    0.2 & 0.2 & 0 & 0.6 \\
    0.3 & 0.3 & 0.4 & 0 \\
    0 & 0 & 0.1 & 0.9 \\
    0 & 0 & 0.8 & 0.2 \\
  \end{pmatrix}.
\]

\begin{enumerate}
  \item (3 points) Split the chain in classes and classify them into closed or not closed.
  \item (2 points) Classify the states into recurrent or transient.
  \item (5 points) A Hedgehog starts in the state one and moves 
  randomly between states according to the transition matrix.

  What is the approximate probability that the Hedgehog will be in the 
  state four after $10^{2021}$ moves?
\end{enumerate}

Note: state number is the row (or column) number.

  \item (10 points) Gleb Zheglov catches one criminal every day. 
  With probability $0.2$ the catched criminal is replaced by $w$ new criminals. 
  Initially there are $n$ criminals in the town. 

  What is the expected time to the ultimate crime eradication in the town?

  \begin{enumerate}
    \item (4 points) Solve the problem for $w=1$ and $k=1$.
    \item (6 points) Solve the problem for arbitrary $w$ and $k$.
  \end{enumerate}

  \item (10 points) The random variables $X_i$ are independend and uniformly distributed on $[0;1]$.
  Find the probability limit
\[
\plim_{n\to\infty}  \max \left\{ \frac{\sum_{i=1}^n X_i}{n}, \frac{2\sum_{i=1}^n X^2_i}{n} \right\}.
\]


\item (10 points) Taxis arrive to the station according to the Poisson process with rate 1 per 5 minutes. 

Let $Y_t$ be the number of taxis that will arrive between 0 and $t$ minutes.

\begin{enumerate}
  \item (2 points) Sketch the expected value of $Y_t$ as a function of $t$.
  \item (8 points) Sketch the probability $\P(Y_t = Y_{60})$ as a function of $t$.
\end{enumerate}

Note: special points like intercepts or extrema should be explicitely marked.

\newpage 
\item (10 points) Prince Myshkin throws a fair coin until two consecutive heads appear. 
Let $N$ be the number of throws. 

Find the moment generating function of $N$. 

Hint: you may use the first step approach.

\item (20 points) Vincenzo Peruggia makes attempts to steal the Mona Lisa painting until the first 
success. 
Each attempt is successful with probability $0.1$.

Let $X$ be the number of attempts and $Z = \min\{X, 5\}$.

\begin{enumerate}
  \item (5 points) How many events are in sigma-algebras $\sigma(Z)$ and $\sigma(X)$?
  \item (5 points) If possible provide an example of events $A$ and $B$ such that: $A\in \sigma(Z)$ but $A\not\in\sigma(X)$; $B\in \sigma(X)$ but $B\not\in\sigma(Z)$.
  \item (10 points) Find $\E(Z \mid X)$ and $\E(X \mid Z)$.
\end{enumerate}






\end{enumerate}


\end{document}

